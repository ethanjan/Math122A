\documentclass[12pt]{article}
 
\usepackage[margin=1in]{geometry}
\usepackage{amsmath,amsthm,amssymb}
\usepackage{mathtools}
\DeclarePairedDelimiter{\ceil}{\lceil}{\rceil}
%\usepackage{mathptmx}
\usepackage{accents}
\usepackage{comment}
\usepackage{graphicx}
\usepackage{IEEEtrantools}
 \usepackage{float}
 
\newcommand{\N}{\mathbb{N}}
\newcommand{\Z}{\mathbb{Z}}
\newcommand{\R}{\mathbb{R}}
\newcommand{\Q}{\mathbb{Q}}
\newcommand*\conj[1]{\bar{#1}}
\newcommand*\mean[1]{\bar{#1}}
\newcommand\widebar[1]{\mathop{\overline{#1}}}


\newcommand{\cc}{{\mathbb C}}
\newcommand{\rr}{{\mathbb R}}
\newcommand{\qq}{{\mathbb Q}}
\newcommand{\nn}{\mathbb N}
\newcommand{\zz}{\mathbb Z}
\newcommand{\aaa}{{\mathcal A}}
\newcommand{\bbb}{{\mathcal B}}
\newcommand{\rrr}{{\mathcal R}}
\newcommand{\fff}{{\mathcal F}}
\newcommand{\ppp}{{\mathcal P}}
\newcommand{\eps}{\varepsilon}
\newcommand{\vv}{{\mathbf v}}
\newcommand{\ww}{{\mathbf w}}
\newcommand{\xx}{{\mathbf x}}
\newcommand{\ds}{\displaystyle}
\newcommand{\Om}{\Omega}
\newcommand{\dd}{\mathop{}\,\mathrm{d}}
\newcommand{\ud}{\, \mathrm{d}}
\newcommand{\seq}[1]{\left\{#1\right\}_{n=1}^\infty}
\newcommand{\isp}[1]{\quad\text{#1}\quad}

\DeclareMathOperator{\imag}{Im}
\DeclareMathOperator{\re}{Re}
\DeclareMathOperator{\diam}{diam}
\DeclareMathOperator{\Tr}{Tr}
\DeclareMathOperator{\cis}{cis}

\def\upint{\mathchoice%
    {\mkern13mu\overline{\vphantom{\intop}\mkern7mu}\mkern-20mu}%
    {\mkern7mu\overline{\vphantom{\intop}\mkern7mu}\mkern-14mu}%
    {\mkern7mu\overline{\vphantom{\intop}\mkern7mu}\mkern-14mu}%
    {\mkern7mu\overline{\vphantom{\intop}\mkern7mu}\mkern-14mu}%
  \int}
\def\lowint{\mkern3mu\underline{\vphantom{\intop}\mkern7mu}\mkern-10mu\int}




\newenvironment{theorem}[2][Theorem]{\begin{trivlist}
\item[\hskip \labelsep {\bfseries #1}\hskip \labelsep {\bfseries #2.}]}{\end{trivlist}}
\newenvironment{lemma}[2][Lemma]{\begin{trivlist}
\item[\hskip \labelsep {\bfseries #1}\hskip \labelsep {\bfseries #2.}]}{\end{trivlist}}
\newenvironment{exercise}[2][Exercise]{\begin{trivlist}
\item[\hskip \labelsep {\bfseries #1}\hskip \labelsep {\bfseries #2.}]}{\end{trivlist}}
\newenvironment{problem}[2][Problem]{\begin{trivlist}
\item[\hskip \labelsep {\bfseries #1}\hskip \labelsep {\bfseries #2.}]}{\end{trivlist}}
\newenvironment{question}[2][Question]{\begin{trivlist}
\item[\hskip \labelsep {\bfseries #1}\hskip \labelsep {\bfseries #2.}]}{\end{trivlist}}
\newenvironment{corollary}[2][Corollary]{\begin{trivlist}
\item[\hskip \labelsep {\bfseries #1}\hskip \labelsep {\bfseries #2.}]}{\end{trivlist}}

\newenvironment{solution}{\begin{proof}[Solution]}{\end{proof}}
 
\begin{document}
 
% --------------------------------------------------------------
%                         Start here
% --------------------------------------------------------------
\title{Math 122A Homework 3}
\author{Ethan Martirosyan}
\date{\today}
\maketitle
\hbadness=99999
\hfuzz=50pt
\section*{Chapter 2 Problem 13}
\subsection*{Part A}
Let $\beta > L$. Then, there must exist some $N$ such that for all $n \geq N$,
\[
\frac{a_{n+1}}{a_n} < \beta
\] This informs us that
\[
a_{N+1} < \beta a_N
\] We also have
\[
a_{N+2} < \beta a_{N+1} < \beta^2 a_N
\] By induction, we may deduce that for any positive integer $p$, we have
\[
a_{N+p} < \beta^p a_N
\] Let $k = N+p$. Then, we have
\[
a_k <  \beta^{k} \cdot (\beta^{-N} a_N)
\] Taking the $k$th root of this, we obtain
\[
\sqrt[k]{a_k} < \beta \cdot \sqrt[k]{\beta^{-N} a_N}
\] Notice that $\beta^{-N} a_N$ is constant, so
\[
\lim_{k\rightarrow \infty} \sqrt[k]{\beta^{-N} a_N} = 1
\] Thus we may deduce that
\[
\varlimsup \sqrt[k]{a_k} \leq \beta
\] Since this is true for every $\beta > L$, we have
\[
\varlimsup \sqrt[k]{a_k} \leq L
\] Next, let $\alpha < L$. Then, there must exist some $N$ such that for all $n \geq N$,
\[
\frac{a_{n+1}}{a_n} > \alpha
\] Thus, we have
\[
a_{N+1} > \alpha a_N
\] Notice that
\[
a_{N+2} > \alpha a_{N+1} > \alpha^2 a_N
\] By induction, we know that for any integer $p > 0$, we have
\[
a_{N+p} > \alpha^p a_N
\] Let $k = N+p$. Then,
\[
a_k > \alpha^k \cdot (\alpha^{-N} a_N)
\] Taking the $k$th root yields
\[
\sqrt[k]{a_k} > \alpha \sqrt[k]{\alpha^{-N} a_N}
\] Since $\alpha^{-N} a_N$ is a constant, we have
\[
\lim_{k \rightarrow \infty} \sqrt[k]{\alpha^{-N} a_N} = 1
\] so that
\[
\varliminf \sqrt[k]{a_k} \geq \alpha
\] Since this is true for every $\alpha \leq L$, we may deduce that
\[
\varliminf \sqrt[k]{a_k} \geq L
\] Thus we have
\[
L \leq \varliminf \sqrt[k]{a_k} \leq \varlimsup \sqrt[k]{a_k} \leq L
\] so that
\[
\lim_{k \rightarrow \infty} \sqrt[k]{a_k} = L
\]
\subsection*{Part B}
Notice that 
\[
\frac{(1/(n+1)!)}{(1/n!)} = \frac{n!}{(n+1)!} = \frac{1}{n+1} 
\] which evidently goes to zero. By Part A, we may conclude that
\[
\bigg(\frac{1}{n!}\bigg)^{1/n} \rightarrow 0
\]
\newpage
\section*{Problem 14}
\subsection*{Part A} Using Part A of $13$, we find that
\[
\frac{1/(n+1)!}{1/n!} = \frac{n!}{(n+1)!} = \frac{1}{n+1}
\] which goes to $0$. Thus the radius of convergence is $\infty$.
\subsection*{Part B}
Since the exponent of $z$ is $2n+1$, we must be careful. Notice that
\[
\bigg \vert \frac{a_{n+1}}{a_n} \bigg \vert = \bigg\vert \frac{1/(2(n+1)+1)!}{1/(2n+1)!} \bigg \vert =  \bigg\vert \frac{(2n+1)!}{(2n+3)!} \bigg \vert = \bigg \vert \frac{1}{(2n+2)(2n+3)} \bigg \vert = 0
\] so that
\[
\varlimsup \bigg \vert \frac{a_{n+1}}{a_n} \bigg \vert = 0
\]
and
\[
\varliminf \bigg \vert \frac{a_{n+1}}{a_n} \bigg \vert = 0
\] (the latter is true because when $n$ is even, $a_n = 0$). Thus we may deduce that
\[
\lim_{n\rightarrow \infty} \bigg \vert \frac{a_{n+1}}{a_n} \bigg \vert = 0
\] so the radius of convergence is $\infty$.
\subsection*{Part C}
Notice that
\[
\frac{(n+1)!/(n+1)^{n+1}}{n!/n^n} = \frac{(n+1)!n^n}{(n+1)^{n+1}n!} = \frac{(n+1)n^n}{(n+1)^{n+1}} = \frac{n^n}{(n+1)^n} = \bigg(\frac{n}{n+1}\bigg)^n
\] Now, we let 
\[
y = \lim_{n\rightarrow \infty} \bigg(\frac{n}{n+1}\bigg)^n
\] and take logarithms to obtain
\[
\ln y = \lim_{n\rightarrow \infty} n \ln \frac{n}{n+1} = \lim_{n\rightarrow \infty} \frac{\ln(n/(n+1))}{1/n}
\] Since the numerator goes to $-\infty$ and the denominator goes to $\infty$ as $n \rightarrow \infty$, we may apply L'Hopital's rule and the chain rule to obtain
\[
\frac{\frac{n+1}{n} \cdot \frac{1}{(n+1)^2} }{-\frac{1}{n^2}} = -\frac{n}{n+1}
\] which goes to $-1$. Thus $\ln y = -1$ and $y = e^{-1}$. Thus the radius of convergence is $e$.
\subsection*{Part D}
We have
\[
\frac{2^{n+1}/(n+1)!}{2^n/n!} = \frac{2}{n+1}
\] which goes to $0$. The radius of convergence is $\infty$.
\newpage
\section*{Chapter 3 Problem 2}
\subsection*{Part A}
The function is $f(z) = x^2 + iy^2$. Notice that $f_x = 2x$ and $f_y = 2yi$. Thus $if_x = 2xi = 2yi = f_y$ if and only if $x = y$. Thus $f$ is differentiable at all points on the line $y=x$.
\subsection*{Part B} 
Let $z$ be an arbitrary point on the line $y=x$. Any neighborhood of $z$ contains points not on the line $y=x$. For these points, the Cauchy-Riemann equation $f_y = if_x$ is not satisfied since $x \neq y$. Thus $f$ is not differentiable at these points, so there does not exist a neighborhood of $z$ on which $f$ is differentiable. Thus $f$ is not analytic at any point $z$ on the line $y=x$. It is also evident that $f$ is not analytic if $y \neq x$ because then $f$ is not even differentiable.
\newpage
\section*{Problem 5}
Notice that $f(z) = f(x,y) = u(x,y) + iv(x,y)$, where $u$ and $v$ are real-valued. Furthermore, we know that $f^\prime(z) = f_x(z) = u_x + iv_x = 0$. Thus $u_x = 0$ and $v_x = 0$ at every point. By the Cauchy-Riemann equations, we have $u_y = 0$ and $v_y = 0$. Now, let $(x_1,y_1)$ and $(x_2,y_2)$ be two points in the region. We claim that $f(x_1,y_1) = f(x_2,y_2)$. Since we are considering a region, we know that there exists some polygonal path between these two points. If $(a,b_1)$ and $(a,b_2)$ are two points on this path with the same $x$ coordinate, we may note that $u(a,b_1) - u(a,b_2) = u_y(c) \cdot (b_2-b_1)$ for some real $c \in (b_1,b_2)$ by the Mean Value Theorem. Since $u_y = 0$, we have $u(a,b_1) = u(a,b_2)$. If $(a_1,b)$ and $(a_2,b)$ are two points on this path with the same $y$ coordinate, we may note that $u(a_1,b) -u(a_2,b) = u_x(c)(a_2  - a_1)$ for some real $c \in (a_1,a_2)$ by the Mean Value Theorem. Since $u_x = 0$, we know that $u(a_1,b) = u(a_2,b)$. Thus, we know that $u$ is constant. Similar reasoning informs us that $v$ is constant. Because polygonal paths only contain horizontal and vertical line segments, we may deduce that $f(x_1,y_1) = u(x_1,y_1) + iv(x_1,y_1) = u(x_2,y_2) + iv(x_2,y_2) = f(x_2,y_2)$. Thus $f$ is constant.
\newpage
\section*{Problem 8}
Since $u(x,y) = x^2 - y^2$, we have $u_x = 2x$ and $u_y = -2y$. Since $u_x = v_y$, we have $v_y = 2x$. Integration in $y$ yields $v(x,y) = 2xy + c$, where $c \in \R$. Now we claim that $f$ is analytic, where $f(x,y) = u(x,y) + i v(x,y) = (x^2 - y^2) + i(2xy+c)$. Notice that $u_x = 2x = v_y$ and $u_y = -2y = -v_x$. Thus $f$ is analytic at all points. In order for $f$ to be analytic, it is necessary and sufficient that $v(x,y) = 2xy + c$.
\newpage
\section*{Problem 9}
Suppose there was an analytic function $f = u + iv$ with $u(x,y) = x^2 + y^2$. Then, we have $u_x = 2x$ and $u_y = 2y$. By the Cauchy-Riemann equations, we have $v_y = u_x = 2x$ and $v_x = -u_y = -2y$. Integrating $v_y = 2x$ in $y$ yields $v(x,y) = 2xy + c$ for some real constant $c$. Integrating $v_x = -2y$ in $x$ yields $v(x,y) = -2xy + d$ for some real constant $d$. Then, we have $2xy + c = -2xy + d$ so that $4xy = c+d$. This is impossible since $4xy$ is not constant when $x$ and $y$ vary. Thus $f$ is not analytic.
\newpage
\section*{Problem 20}
Notice that
\[
\sin x \cosh y + i \cos x \sinh y = \frac{e^{ix} - e^{-ix}}{2i} \cdot \frac{e^{y}+e^{-y}}{2} + i \cdot \frac{e^{ix} + e^{-ix}}{2} \cdot \frac{e^y - e^{-y}}{2} 
\] is equal to
\[
-i\bigg(\frac{e^{ix+y} - e^{-ix+y} + e^{ix-y} - e^{-ix-y}}{4}\bigg) + i\bigg(\frac{e^{ix+y} + e^{-ix+y} - e^{ix - y} - e^{-ix-y}}{4} \bigg)
\] which is equal to
\[
i\bigg(\frac{e^{-ix+y} - e^{ix-y}}{2}\bigg)
\] This is equal to
\[
\frac{e^{ix-y}-e^{-ix+y}}{2i}
\] which is equal to
\[
\frac{1}{2i}(e^{i(x+yi)} - e^{-i(x+yi)}) = \sin(x+iy)
\]
\end{document} 