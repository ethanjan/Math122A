\documentclass[12pt]{article}
 
\usepackage[margin=1in]{geometry}
\usepackage{amsmath,amsthm,amssymb}
\usepackage{mathtools}
\DeclarePairedDelimiter{\ceil}{\lceil}{\rceil}
%\usepackage{mathptmx}
\usepackage{accents}
\usepackage{comment}
\usepackage{graphicx}
\usepackage{IEEEtrantools}
 \usepackage{float}
 
\newcommand{\N}{\mathbb{N}}
\newcommand{\Z}{\mathbb{Z}}
\newcommand{\R}{\mathbb{R}}
\newcommand{\Q}{\mathbb{Q}}
\newcommand*\conj[1]{\bar{#1}}
\newcommand*\mean[1]{\bar{#1}}
\newcommand\widebar[1]{\mathop{\overline{#1}}}


\newcommand{\cc}{{\mathbb C}}
\newcommand{\rr}{{\mathbb R}}
\newcommand{\qq}{{\mathbb Q}}
\newcommand{\nn}{\mathbb N}
\newcommand{\zz}{\mathbb Z}
\newcommand{\aaa}{{\mathcal A}}
\newcommand{\bbb}{{\mathcal B}}
\newcommand{\rrr}{{\mathcal R}}
\newcommand{\fff}{{\mathcal F}}
\newcommand{\ppp}{{\mathcal P}}
\newcommand{\eps}{\varepsilon}
\newcommand{\vv}{{\mathbf v}}
\newcommand{\ww}{{\mathbf w}}
\newcommand{\xx}{{\mathbf x}}
\newcommand{\ds}{\displaystyle}
\newcommand{\Om}{\Omega}
\newcommand{\dd}{\mathop{}\,\mathrm{d}}
\newcommand{\ud}{\, \mathrm{d}}
\newcommand{\seq}[1]{\left\{#1\right\}_{n=1}^\infty}
\newcommand{\isp}[1]{\quad\text{#1}\quad}
\newcommand*\diff{\mathop{}\!\mathrm{d}}

\DeclareMathOperator{\imag}{Im}
\DeclareMathOperator{\re}{Re}
\DeclareMathOperator{\diam}{diam}
\DeclareMathOperator{\Tr}{Tr}
\DeclareMathOperator{\cis}{cis}

\def\upint{\mathchoice%
    {\mkern13mu\overline{\vphantom{\intop}\mkern7mu}\mkern-20mu}%
    {\mkern7mu\overline{\vphantom{\intop}\mkern7mu}\mkern-14mu}%
    {\mkern7mu\overline{\vphantom{\intop}\mkern7mu}\mkern-14mu}%
    {\mkern7mu\overline{\vphantom{\intop}\mkern7mu}\mkern-14mu}%
  \int}
\def\lowint{\mkern3mu\underline{\vphantom{\intop}\mkern7mu}\mkern-10mu\int}




\newenvironment{theorem}[2][Theorem]{\begin{trivlist}
\item[\hskip \labelsep {\bfseries #1}\hskip \labelsep {\bfseries #2.}]}{\end{trivlist}}
\newenvironment{lemma}[2][Lemma]{\begin{trivlist}
\item[\hskip \labelsep {\bfseries #1}\hskip \labelsep {\bfseries #2.}]}{\end{trivlist}}
\newenvironment{exercise}[2][Exercise]{\begin{trivlist}
\item[\hskip \labelsep {\bfseries #1}\hskip \labelsep {\bfseries #2.}]}{\end{trivlist}}
\newenvironment{problem}[2][Problem]{\begin{trivlist}
\item[\hskip \labelsep {\bfseries #1}\hskip \labelsep {\bfseries #2.}]}{\end{trivlist}}
\newenvironment{question}[2][Question]{\begin{trivlist}
\item[\hskip \labelsep {\bfseries #1}\hskip \labelsep {\bfseries #2.}]}{\end{trivlist}}
\newenvironment{corollary}[2][Corollary]{\begin{trivlist}
\item[\hskip \labelsep {\bfseries #1}\hskip \labelsep {\bfseries #2.}]}{\end{trivlist}}

\newenvironment{solution}{\begin{proof}[Solution]}{\end{proof}}
 
\begin{document}
 
% --------------------------------------------------------------
%                         Start here
% --------------------------------------------------------------
\title{Math 122A Homework 6}
\author{Ethan Martirosyan}
\date{\today}
\maketitle
\hbadness=99999
\hfuzz=50pt
\section*{Chapter 5}
\subsection*{Problem 1}
To find the power series expansion of $f(z) = z^2$ around $z = 2$, we note that
\[
f(z) = f(2) + f^\prime(2)(z-2) + \frac{f^{\prime\prime}(2)}{2!}(z-2)^2 + \cdots 
\] Noting that $f^\prime(z) = 2z$ and $f^{\prime\prime}(z) = 2$, we obtain
\[
f(z) = 2^2 + 2(2)(z-2) + \frac{2}{2}(z-2)^2 = 4 + 4(z-2) + (z-2)^2
\]
\newpage
\subsection*{Problem 2}
To find the power series expansion for $f(z) = e^z$ about any point $a$, we may write
\[
e^z = \sum_{k=0}^\infty \frac{f^{(k)}(a)}{k!}(z-a)^k
\] Since $f^{(k)}(z) = f(z)$, we find that
\[
\sum_{k=0}^\infty \frac{f^{(k)}(a)}{k!}(z-a)^k = \sum_{k=0}^\infty \frac{e^a}{k!}(z-a)^k = e^a \sum_{k=0}^\infty \frac{(z-a)^k}{k!}
\]
\newpage
\subsection*{Problem 3}
\subsubsection*{Part A}
In order to show that an odd entire function $f$ has only odd terms in its power series expansion about $0$, we must show that $f^{(k)}(0) = 0$ when $k$ is even. First, notice that since $f$ is odd, we have $f(0) = -f(-0) = -f(0)$ so that $2f(0) = 0$ and $f(0) = 0$. Thus, we may deduce that any odd function must take the value $0$ at the point $0$. Therefore, we must show that $f^{k}$ is odd when $k$ is even. We know that $f^{(0)} = f$ is odd. Next, we claim that $f^{(1)} = f^\prime$ is even. Since $f$ is odd, we have $f(z) = -f(-z)$. Taking derivatives, we obtain $f^\prime(z) = f^\prime(-z)$, so that $f^\prime$ is even. Next, we note that $f^{\prime\prime}(z) = -f^{\prime\prime}(-z)$, so that $f^{(2)} = f^{\prime\prime}$ is odd. Continuing inductively, we may deduce that $f^k$ is odd when $k$ is even. Therefore, we find that $f^{k}(0)=0$ for even $k$ so that $f$ has only odd terms in its power series expansion about $0$.
\subsubsection*{Part B} 
We claim that any even function $f$ has only even terms in its power series expansion about $0$. To prove this, we must show that $f^{(k)}(0) = 0$ when $k$ is odd. Since $f$ is even, we have $f(z) = f(-z)$ for all $z \in \mathbb{C}$. Taking derivatives yields $f^\prime(z) = -f^\prime(-z)$ so that $f^\prime$ is odd. As above, we know that $f^{(1)}(0) = f^\prime(0) = 0$. Next, we find that $f^{\prime\prime}(z) = f^{\prime\prime}(-z)$ so that $f^{(2)}$ is even. Continuing inductively, we find that $f^{(k)}$ is odd for all odd $k$ so that $f^{(k)}(0) = 0$ for all odd $k$. This shows that $f$ has only even terms in its power series expansion about $0$.
\newpage
\subsection*{Problem 4}
In the proof of Theorem $5.5$, it was established that
\[
f(z) = \sum_{k=0}^\infty \Bigg( \frac{1}{2\pi i }  \int_C \frac{f(\omega)}{\omega^{k+1}} \diff \omega \Bigg) z^k
\] By Corollary $2.11$, we have
\[
f(z) = \sum_{k=0}^\infty \frac{f^{(k)}(0)}{k!} z^k
\] Comparing coefficients, we find that
\[
\frac{1}{2\pi i } \int_C \frac{f(\omega)}{\omega^{k+1}} \diff \omega = \frac{f^{(k)}(0)}{k!}
\] so that
\[
f^{(k)}(0) = \frac{k!}{2\pi i} \int_C \frac{f(\omega)}{\omega^{k+1}} \diff \omega
\]
\newpage
\subsection*{Problem 5}
Define $h$ as follows:
\[
h(z) = f(z+a)
\] Notice that
\[
h^{(k)}(z) = f^{(k)}(z+a)
\] for every positive integer $k$. Now, we note that
\[
f^{(k)}(a) = h^{(k)}(0) = \frac{k!}{2\pi i} \int_C \frac{h(\omega)}{\omega^{k+1}} \diff \omega = \frac{k!}{2\pi i} \int_C \frac{f(\omega+a)}{\omega^{k+1}} \diff \omega
\] Letting $s = \omega+a$, we find that
\[
\frac{k!}{2\pi i} \int_C \frac{f(\omega+a)}{\omega^{k+1}} \diff \omega = \frac{k!}{2\pi i} \int_C \frac{f(s)}{(s-a)^{k+1}} \diff s
\] so that
\[
f^{(k)}(a) = \frac{k!}{2\pi i} \int_C \frac{f(s)}{(s-a)^{k+1}} \diff s
\]
\newpage
\subsection*{Problem 6}
\subsubsection*{Part A}
By Problem $4$, we have
\[
C_k = \frac{f^{(k)}(0)}{k!} = \frac{1}{2\pi i } \int_C \frac{f(\omega)}{\omega^{k+1}} \diff \omega
\] Notice that the length of the path of integration is $2\pi R$ (since $C$ is a circle with radius $R$) and that
\[
\bigg \vert \frac{f(\omega)}{\omega^{k+1}} \bigg \vert \leq \frac{M}{R^{k+1}}
\] Thus, we realize that
\[
\vert C_k \vert = \bigg \vert \frac{1}{2\pi i} \int_C \frac{f(\omega)}{\omega^{k+1}} \diff \omega \bigg \vert \leq \frac{1}{2 \pi} \cdot 2 \pi R \cdot \frac{M}{R^{k+1}} = \frac{M}{R^k} 
\]
\subsubsection*{Part B}
We suppose that an arbitrary polynomial $P(z)$ is bounded by $1$ in the unit disk. Since $P(z)$ is continuous, it must also be bounded by $1$ on the unit disk. Thus, we may appeal to Part A of Problem $6$ with $M = 1$ and $R = 1$ to deduce that
\[
\vert C_k \vert \leq \frac{M}{R^k} = \frac{1}{1^k} = 1
\] where $C_k$ is the $k$th coefficient of the polynomial $P(z)$. That is, all the coefficients of the polynomial are bounded by $1$.
\newpage
\subsection*{Problem 7}
Let $R > 0$ be an arbitrary positive number. On the circle $\vert z \vert = R$, we know that $f(z)$ is bounded by $A+B \vert z \vert^k = A + B R^k$. Appealing to Part A of Problem $6$, we note that for any $j > k$, we have
\[
\vert C_j \vert \leq \frac{A + B R^k}{R^j} = \frac{A}{R^j} + \frac{B}{R^{j-k}}
\] Since this is true for every $R > 0$, we may take the limit as $R$ approaches $\infty$ in order to deduce that $C_j = 0$ for all $j > k$.
\newpage
\subsection*{Problem 8}
We claim that $C_k = 0$ for $k \geq 2$, where $C_k$ is the $k$th coefficient of the power series of $f$ centered at $0$. Let $R > 0$. On the circle $\vert z \vert = R$, we know that $f(z)$ is bounded by $A + B \vert z \vert^{3/2} = A + BR^{3/2}$. By Part A of Problem $6$, we have
\[
\vert C_k \vert \leq \frac{A+BR^{3/2}}{R^k} = \frac{A}{R^k} + BR^{3/2-k}
\] As $R \rightarrow \infty$, it is evident that $\frac{A}{R^k} + BR^{3/2-k} \rightarrow 0$ as long as $k \geq 2$. Thus, we may deduce that $f$ is a linear polynomial.
\newpage
\subsection*{Problem 9}
Since $\vert f^\prime(z) \vert \leq \vert z \vert$, we may deduce from the Extended Liouville Theorem that $f^\prime$ is a polynomial of degree $1$. Thus, $f$ is a polynomial of degree $2$. Now, we may write
\[
f(z) = f(0) + f^\prime(0)z + \frac{f^{\prime\prime}(0)}{2}z^2
\] Since $\vert f^\prime(0) \vert \leq 0$, we know that $f^\prime(0) = 0$. Now, we note that
\[
f(z) = f(0) + \frac{f^{\prime\prime}(0)}{2} z^2
\] Furthermore, we have
\[
 f^{\prime\prime}(0) = \frac{1}{2\pi i} \int_C \frac{f^\prime(z)}{z} \diff z
\] where $C$ is the unit circle. Notice that $C$ has circumference $2\pi$ and $\vert f^\prime(z)/z \vert \leq 1$ so that
\[
\vert f^{\prime\prime}(0) \vert = \frac{1}{2\pi}  \bigg \vert \int_C  \frac{f^\prime(z)}{z}  \diff z \bigg \vert \leq 1
\] Thus, letting $a = f(0)$ and $b = f^{\prime\prime}(0)/2$, we may conclude that
\[
f(z) = a + bz^2
\] where $\vert b \vert \leq \frac{1}{2}$.
\end{document} 