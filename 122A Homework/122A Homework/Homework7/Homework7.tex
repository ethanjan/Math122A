\documentclass[12pt]{article}
 
\usepackage[margin=1in]{geometry}
\usepackage{amsmath,amsthm,amssymb}
\usepackage{mathtools}
\DeclarePairedDelimiter{\ceil}{\lceil}{\rceil}
%\usepackage{mathptmx}
\usepackage{accents}
\usepackage{comment}
\usepackage{graphicx}
\usepackage{IEEEtrantools}
 \usepackage{float}
 
\newcommand{\N}{\mathbb{N}}
\newcommand{\Z}{\mathbb{Z}}
\newcommand{\R}{\mathbb{R}}
\newcommand{\Q}{\mathbb{Q}}
\newcommand*\conj[1]{\bar{#1}}
\newcommand*\mean[1]{\bar{#1}}
\newcommand\widebar[1]{\mathop{\overline{#1}}}


\newcommand{\cc}{{\mathbb C}}
\newcommand{\rr}{{\mathbb R}}
\newcommand{\qq}{{\mathbb Q}}
\newcommand{\nn}{\mathbb N}
\newcommand{\zz}{\mathbb Z}
\newcommand{\aaa}{{\mathcal A}}
\newcommand{\bbb}{{\mathcal B}}
\newcommand{\rrr}{{\mathcal R}}
\newcommand{\fff}{{\mathcal F}}
\newcommand{\ppp}{{\mathcal P}}
\newcommand{\eps}{\varepsilon}
\newcommand{\vv}{{\mathbf v}}
\newcommand{\ww}{{\mathbf w}}
\newcommand{\xx}{{\mathbf x}}
\newcommand{\ds}{\displaystyle}
\newcommand{\Om}{\Omega}
\newcommand{\dd}{\mathop{}\,\mathrm{d}}
\newcommand{\ud}{\, \mathrm{d}}
\newcommand{\seq}[1]{\left\{#1\right\}_{n=1}^\infty}
\newcommand{\isp}[1]{\quad\text{#1}\quad}
\newcommand*\diff{\mathop{}\!\mathrm{d}}

\DeclareMathOperator{\imag}{Im}
\DeclareMathOperator{\re}{Re}
\DeclareMathOperator{\diam}{diam}
\DeclareMathOperator{\Tr}{Tr}

\def\upint{\mathchoice%
    {\mkern13mu\overline{\vphantom{\intop}\mkern7mu}\mkern-20mu}%
    {\mkern7mu\overline{\vphantom{\intop}\mkern7mu}\mkern-14mu}%
    {\mkern7mu\overline{\vphantom{\intop}\mkern7mu}\mkern-14mu}%
    {\mkern7mu\overline{\vphantom{\intop}\mkern7mu}\mkern-14mu}%
  \int}
\def\lowint{\mkern3mu\underline{\vphantom{\intop}\mkern7mu}\mkern-10mu\int}




\newenvironment{theorem}[2][Theorem]{\begin{trivlist}
\item[\hskip \labelsep {\bfseries #1}\hskip \labelsep {\bfseries #2.}]}{\end{trivlist}}
\newenvironment{lemma}[2][Lemma]{\begin{trivlist}
\item[\hskip \labelsep {\bfseries #1}\hskip \labelsep {\bfseries #2.}]}{\end{trivlist}}
\newenvironment{exercise}[2][Exercise]{\begin{trivlist}
\item[\hskip \labelsep {\bfseries #1}\hskip \labelsep {\bfseries #2.}]}{\end{trivlist}}
\newenvironment{problem}[2][Problem]{\begin{trivlist}
\item[\hskip \labelsep {\bfseries #1}\hskip \labelsep {\bfseries #2.}]}{\end{trivlist}}
\newenvironment{question}[2][Question]{\begin{trivlist}
\item[\hskip \labelsep {\bfseries #1}\hskip \labelsep {\bfseries #2.}]}{\end{trivlist}}
\newenvironment{corollary}[2][Corollary]{\begin{trivlist}
\item[\hskip \labelsep {\bfseries #1}\hskip \labelsep {\bfseries #2.}]}{\end{trivlist}}

\newenvironment{solution}{\begin{proof}[Solution]}{\end{proof}}
 
\begin{document}
 
% --------------------------------------------------------------
%                         Start here
% --------------------------------------------------------------
\title{Math 122A Homework 7}
\author{Ethan Martirosyan}
\date{\today}
\maketitle
\hbadness=99999
\hfuzz=50pt
\section*{Bak Newmann Chapter 5}
\subsection*{Problem 17}
First, let us consider the case $n = 2$. Then, we note that
\[
a_1z_1 + a_2 z_2 = a_1 z_1 + (1-a_1)z_2
\] which is in the convex set $S$ by the definition of convex set. Next, we may suppose that $n\geq 2$ and that
\[
a_1 z_1 + \cdots + a_n z_n
\] is in $S$ whenever $a_i \geq 0$ for all $i$ and $\sum a_i = 1$. We must show that
\[
a_1 z_1 + \cdots + a_{n+1} z_{n+1}
\] is in $S$ whenever $a_i \geq 0$ for all $i$ and $\sum a_i = 1$. Notice that
\begin{align*}
a_1 z_1 + \cdots + a_{n+1}z_{n+1} = (1-a_{n+1})\Bigg(\frac{a_1}{1-a_{n+1}}z_1 + \cdots + \frac{a_n}{1-a_{n+1}}z_n\Bigg) + a_{n+1}z_{n+1}
\end{align*} By our induction hypothesis, we know that
\[
\frac{a_1}{1-a_{n+1}}z_1 + \cdots + \frac{a_n}{1-a_{n+1}}z_n
\] is in $S$. By the definition of convex set, we may deduce that
\[
(1-a_{n+1})\Bigg(\frac{a_1}{1-a_{n+1}}z_1 + \cdots + \frac{a_n}{1-a_{n+1}}z_n\Bigg) + a_{n+1}z_{n+1}
\] is in $S$, thus completing the proof.
\newpage
\subsection*{Problem 19}
Notice that an antiderivative of
\[
P(z) = 1 + 2z + \cdots nz^{n-1}
\] is
\[
Q(z) = 1 + z + \cdots + z^n
\] Using the formula for geometric sums, we obtain
\[
Q(z) =  1 + z + \cdots + z^n = \frac{z^{n+1}-1}{z-1}
\] Notice that the zeroes of the latter expression are
\[
e^\frac{2\pi i a}{n+1}
\] where $1 \leq a \leq n$. That is, all the zeroes of $Q(z)$ lie on the unit circle. Thus the convex hull of these solutions must lie inside the unit circle. By the Gauss-Lucas Theorem, we may deduce that all the zeroes of $P(z)$ lie inside the unit circle.
\newpage
\section*{Bak Newmann Chapter 6}
\subsection*{Problem 1}
Let $\beta = 1+i$. Then, we have
\[
\frac{1}{z} = \frac{1}{\beta+(z-\beta)}
\] Factoring out $\beta$ yields
\[
\frac{1}{\beta+(z-\beta)} = \frac{1}{\beta(1+\frac{z-\beta}{\beta})} = \frac{1}{\beta} \cdot \frac{1}{1 - (-\frac{z-\beta}{\beta})}
\] Notice that
\[
\frac{1}{\beta} \cdot \frac{1}{1 - (-\frac{z-\beta}{\beta})} = \frac{1}{\beta} \bigg(1 - \frac{z - \beta}{\beta} + \frac{(z - \beta)^2}{
\beta^2} + \cdots \bigg)
\] This is equal to
\[
\frac{1}{\beta} \sum_{n=0}^\infty \frac{(-1)^n (z-\beta)^n}{\beta^n} = \sum_{n=0}^\infty \frac{(-1)^n (z-\beta)^n}{\beta^{n+1}}
\] Substituting $\beta = 1+i$, we obtain
\[
\frac{1}{z} = \sum_{n=0}^\infty \frac{(-1)^n (z-1-i)^n}{(1+i)^{n+1}}
\]
\newpage
\subsection*{Problem 2}
Notice that $1-z-2z^2 = (1-2z)(1+z)$. Now, we may write
\[
\frac{1}{1-z-2z^2} = \frac{A}{1-2z} + \frac{B}{1+z}
\] where $A$ and $B$ are constants to be determined. Multiplying both sides by $1-z-2z^2$ yields
\[
1 = A(1+z) + B(1-2z) 
\] so that
\[
1 = A + Az + B - 2Bz = A+B + (A-2B)z
\] Thus, we know that $A+B = 1$ and $A-2B = 0$. Since $A = 2B$, we obtain $2B + B = 3B =1$ so that $B = 1/3$ and $A = 1 - 1/3 = 2/3$. We have
\begin{align*}
&\frac{1}{1-z-2z^2} = \frac{2/3}{1-2z} + \frac{1/3}{1+z} = \frac{2}{3}\sum_{n=0}^\infty 2^n z^n + \frac{1}{3}\sum_{n=0}^\infty (-1)^n z^n = \frac{1}{3}\bigg(\sum_{n=0}^\infty 2^{n+1} z^n + \sum_{n=0}^\infty  (-1)^n z^n \bigg)\\
& = \frac{1}{3}\bigg(\sum_{n=0}^\infty (2^{n+1}+(-1)^n)z^n\bigg) = \sum_{n=0}^\infty \frac{2^{n+1}+(-1)^n}{3}z^n
\end{align*}
\newpage
\subsection*{Problem 3}
Differentiating the identity
\[
1+z+z^2+\cdots = \frac{1}{1-z}
\] yields
\[
1 + 2z + \cdots = \frac{1}{(1-z)^2}
\] Multiplying both sides by $z$, we obtain
\[
z + 2z^2 + \cdots = \frac{z}{(1-z)^2}
\] That is, we have established the following identity:
\[
\sum nz^n = \frac{z}{(1-z)^2}
\] Differentiating this identity yields
\[
\sum n^2 z^{n-1} = \frac{1}{(1-z)^2} + \frac{2z}{(1-z)^3} = \frac{1-z}{(1-z)^3} + \frac{2z}{(1-z)^3} =\frac{1+ z}{(1-z)^3}
\] Multiplying by $z$, we obtain
\[
\sum n^2 z^n  = \frac{z(1+z)}{(1-z)^3}
\]
\newpage
\subsection*{Problem 4}
For the sake of contradiction, suppose that for every positive integer $n$, we have
\[
f(n) = \frac{1}{n+1} = \frac{1/n}{1/n + 1}
\] Notice that the functions $f$ and $g(z) = z/(z+1)$ agree at all points in the set $S = \{1/n: n \in \N\}$. This set has an accumulation point at $0$. By Corollary $6.10$ in Bak and Newman, we may deduce that $f = g$ throughout the region $\vert z \vert < 1$. By continuity, we know that $f = g$ on $\vert z \vert = 1$ except when $z = -1$. No matter how $f(-1)$ is defined, we know that $f$ is not continuous at $-1$ since
\[
\lim_{z \rightarrow -1} f(z) = \lim_{z \rightarrow -1} \frac{z}{z+1} = \infty
\] Thus $f$ is not analytic at $-1$, contrary to our assumptions about $f$.
\newpage
\section*{Needham Chapter 9}
\subsection*{Problem 1}
Let us rewrite the integral 
\[
\int_0^{2\pi} \frac{\diff t}{1 + a^2 - 2a \cos t }
\] as a contour integral around the unit circle $C$ by introducing the substitutions $\cos t = \frac{1}{2}( z + 1/z)$ and $\diff z = iz \diff t$, as the author does on page $437$ of Visual Complex Analysis. Then, we find that
\[
\int_0^{2\pi} \frac{\diff t}{1 + a^2 - 2a \cos t } = \oint_C \frac{-i (\diff z /z)}{1+a^2 - a(z+1/z)}
\] Manipulating the integrand, we find that
\begin{align*}
\frac{-i (\diff z /z)}{1+a^2 - a(z+1/z)} &= \frac{i\diff z }{-z - za^2 + az^2 + a} = \frac{i \diff z}{az^2 - za^2 - z + a} = \frac{i \diff z}{az(z - a) - (z - a)} \\
& = \frac{i \diff z}{(z - a)(az-1)}
\end{align*} Thus, we may conclude that
\[
\int_0^{2\pi} \frac{\diff t}{1 + a^2 - 2a \cos t } = \oint_C \frac{i \diff z}{(z - a)(az-1)}
\] Next, we claim that
\[
\oint_C \frac{i \diff z}{(z - a)(az-1)} = \frac{2\pi}{1-a^2}
\] if $0 < a < 1$. Let $f(z) = i/(az-1)$. Notice that $f$ is analytic on the unit disk since its only singularity $z = \frac{1}{a}$ has modulus strictly greater than $1$. Appealing to the Cauchy Integral Formula, we find that
\[
\oint_C \frac{i \diff z}{(z - a)(az-1)} =  \oint_C \frac{f(z)}{z-a}\diff z = 2\pi i \cdot f(a) = 2\pi i \cdot \frac{i}{a^2 - 1} = \frac{-2\pi}{a^2 - 1} = \frac{2\pi}{1-a^2}
\] We may deduce that
\[
\int_0^{2\pi} \frac{\diff t}{1 + a^2 - 2a \cos t } = \frac{2\pi}{1-a^2}
\] for $0 < a < 1$.
\newpage
\subsection*{Problem 3}
In Visual Complex Analysis, it is shown that
\[
f^{(r)}(0) = \frac{r!}{2\pi i} \oint_C \frac{f(z)}{z^{r+1}} \diff z
\] for any entire function $f$, any nonnegative integer $r$, and any closed loop $C$. Let $f(z) = (1+z)^n$. Then, we have
\[
f^{(r)}(0) = \frac{r!}{2\pi i} \oint_C \frac{(1+z)^n}{z^{r+1}} \diff z
\] By the definition of $f$, we know that
\[
f^\prime(z) = n(1+z)^{n-1}, \; f^{\prime\prime}(z) = n(n-1)(1+z)^{n-2}, \; f^{(3)}(z) = n(n-1)(n-2)(1+z)^{n-3},\ldots
\] By induction, we may deduce that
\[
f^{(r)}(z) = n(n-1)\cdots(n-r+1) (1+z)^{n-r}
\] Setting $z = 0$, we obtain
\[
f^{(r)}(0) = n(n-1)\cdots(n-r+1) = \frac{n!}{(n-r)!}
\] We have
\[
\frac{n!}{(n-r)!} = f^{(r)}(0) = \frac{r!}{2\pi i} \oint_C \frac{(1+z)^n}{z^{r+1}} \diff z
\] Dividing both sides by $r!$ yields
\[
\frac{n!}{r!(n-r)!} = \frac{1}{2\pi i} \oint_C \frac{(1+z)^n}{z^{r+1}} \diff z
\] or
\[
\binom{n}{r} = \frac{1}{2\pi i} \oint_C \frac{(1+z)^n}{z^{r+1}} \diff z
\] Now, we may suppose that $C$ is the unit circle, and we will use the above formula to prove that
\[
\binom{2n}{n} \leq 4^n
\] We have
\[
\binom{2n}{n} = \frac{1}{2\pi i} \oint_C \frac{(1+z)^{2n}}{z^{n+1}} \diff z \leq \frac{1}{2\pi} \oint_C \bigg \vert \frac{(1+z)^{2n}}{z^{n+1}} \bigg \vert \diff z 
\] Notice that
\[
\vert (1+z)^{2n} \vert = \vert 1+z \vert^{2n} \leq 2^{2n} = 4^n
\] and
\[
\vert z^{n+1} \vert = \vert z \vert^{n+1} = 1
\] for all $z$ on the unit circle $C$. Furthermore, the circumference of $C$ is $2\pi$. Thus, using the $M$-$L$ formula from Bak and Newman, we find that
\[
\frac{1}{2\pi} \oint_C \bigg \vert \frac{(1+z)^{2n}}{z^{n+1}} \bigg \vert \diff z \leq \frac{1}{2\pi} \cdot 2\pi \cdot 4^n = 4^n
\] so that
\[
\binom{2n}{n} \leq 4^n
\]
\end{document} 