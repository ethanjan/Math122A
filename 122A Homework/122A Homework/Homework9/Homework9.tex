\documentclass[12pt]{article}
 
\usepackage[margin=1in]{geometry}
\usepackage{amsmath,amsthm,amssymb}
\usepackage{mathtools}
\DeclarePairedDelimiter{\ceil}{\lceil}{\rceil}
%\usepackage{mathptmx}
\usepackage{accents}
\usepackage{comment}
\usepackage{graphicx}
\usepackage{IEEEtrantools}
 \usepackage{float}
 
\newcommand{\N}{\mathbb{N}}
\newcommand{\Z}{\mathbb{Z}}
\newcommand{\R}{\mathbb{R}}
\newcommand{\Q}{\mathbb{Q}}
\newcommand*\conj[1]{\bar{#1}}
\newcommand*\mean[1]{\bar{#1}}
\newcommand\widebar[1]{\mathop{\overline{#1}}}


\newcommand{\cc}{{\mathbb C}}
\newcommand{\rr}{{\mathbb R}}
\newcommand{\qq}{{\mathbb Q}}
\newcommand{\nn}{\mathbb N}
\newcommand{\zz}{\mathbb Z}
\newcommand{\aaa}{{\mathcal A}}
\newcommand{\bbb}{{\mathcal B}}
\newcommand{\rrr}{{\mathcal R}}
\newcommand{\fff}{{\mathcal F}}
\newcommand{\ppp}{{\mathcal P}}
\newcommand{\eps}{\varepsilon}
\newcommand{\vv}{{\mathbf v}}
\newcommand{\ww}{{\mathbf w}}
\newcommand{\xx}{{\mathbf x}}
\newcommand{\ds}{\displaystyle}
\newcommand{\Om}{\Omega}
\newcommand{\dd}{\mathop{}\,\mathrm{d}}
\newcommand{\ud}{\, \mathrm{d}}
\newcommand{\seq}[1]{\left\{#1\right\}_{n=1}^\infty}
\newcommand{\isp}[1]{\quad\text{#1}\quad}
\newcommand*\diff{\mathop{}\!\mathrm{d}}

\DeclareMathOperator{\imag}{Im}
\DeclareMathOperator{\re}{Re}
\DeclareMathOperator{\diam}{diam}
\DeclareMathOperator{\Tr}{Tr}

\def\upint{\mathchoice%
    {\mkern13mu\overline{\vphantom{\intop}\mkern7mu}\mkern-20mu}%
    {\mkern7mu\overline{\vphantom{\intop}\mkern7mu}\mkern-14mu}%
    {\mkern7mu\overline{\vphantom{\intop}\mkern7mu}\mkern-14mu}%
    {\mkern7mu\overline{\vphantom{\intop}\mkern7mu}\mkern-14mu}%
  \int}
\def\lowint{\mkern3mu\underline{\vphantom{\intop}\mkern7mu}\mkern-10mu\int}




\newenvironment{theorem}[2][Theorem]{\begin{trivlist}
\item[\hskip \labelsep {\bfseries #1}\hskip \labelsep {\bfseries #2.}]}{\end{trivlist}}
\newenvironment{lemma}[2][Lemma]{\begin{trivlist}
\item[\hskip \labelsep {\bfseries #1}\hskip \labelsep {\bfseries #2.}]}{\end{trivlist}}
\newenvironment{exercise}[2][Exercise]{\begin{trivlist}
\item[\hskip \labelsep {\bfseries #1}\hskip \labelsep {\bfseries #2.}]}{\end{trivlist}}
\newenvironment{problem}[2][Problem]{\begin{trivlist}
\item[\hskip \labelsep {\bfseries #1}\hskip \labelsep {\bfseries #2.}]}{\end{trivlist}}
\newenvironment{question}[2][Question]{\begin{trivlist}
\item[\hskip \labelsep {\bfseries #1}\hskip \labelsep {\bfseries #2.}]}{\end{trivlist}}
\newenvironment{corollary}[2][Corollary]{\begin{trivlist}
\item[\hskip \labelsep {\bfseries #1}\hskip \labelsep {\bfseries #2.}]}{\end{trivlist}}

\newenvironment{solution}{\begin{proof}[Solution]}{\end{proof}}
 
\begin{document}
 
% --------------------------------------------------------------
%                         Start here
% --------------------------------------------------------------
\title{Math 122A Homework 9}
\author{Ethan Martirosyan}
\date{\today}
\maketitle
\hbadness=99999
\hfuzz=50pt
\section*{Chapter 8}
\subsection*{Problem 1}
Let $z \in \cc \setminus S$, and let us consider the curve $\gamma(t) = tz + (1-t)\alpha$ for $t \geq 1$. Notice that $\gamma(1) = z$ and that 
\[
\lim_{t\rightarrow \infty} \gamma(t) = \infty
\] since $\gamma(t) = \alpha + t(z-\alpha)$ and $z \neq \alpha$. Finally, we claim that $\gamma(t) \in \cc \setminus S$ for all $t \geq 1$. For the sake of contradiction, suppose that there exists some $t\geq 1$ such that $\omega = \gamma(t) \in S$. By the definition of $S$ as a star-like region, we know that $S$ contains the entire line segment between $\alpha$ and $\omega$. In particular, we know that $\gamma(1) = z \in S$ (since $\gamma$ is a line), contradicting our assumption that $z \in \cc \setminus S$. Thus, we may deduce that $\gamma(t) \in \cc \setminus S$ for all $t \geq 1$. By definition, we may conclude that $S$ is simply connected.
\newpage
\subsection*{Problem 2}
We claim that if $C$ is a convex region, then $C$ is star-like. Let $z_1$ be an arbitrary point in $C$. By the definition of convexity, we know that for any other point $z \in C$, the region $C$ contains the line-segment between $z_1$ and $z$. That is, we have found some point $z_1 \in C$ such that $C$ contains the line segment between $z_1$ and $z$ for all $z \in C$. By definition, $C$ is a star-like region. By Problem $1$, we know that $C$ is simply connected.
\newpage
\section*{Chapter 9}
\subsection*{Problem 1}
First, we show that the singularity at $z_0$ is not removable. For the sake of contradiction, suppose that the singularity at $z_0$ is removable. Then there must exist some function $g$ analytic at $z_0$ such that $f(z) = g(z)$ for all $z$ in some deleted neighborhood of $z_0$. Then, we find that
\[
\lim_{z \rightarrow z_0} f(z) = \lim_{z \rightarrow z_0} g(z) = g(z_0)
\] which contradicts the fact that $f(z) \rightarrow \infty$ as $z \rightarrow z_0$. Next, we claim that the singularity at $z_0$ is not essential. For the sake of contradiction, suppose that the singularity at $z_0$ is essential. Since we are assuming that $f(z) \rightarrow \infty$ as $z \rightarrow z_0$, there must exist some $\delta > 0$ such that $0 < \vert z - z_0 \vert < \delta$ implies that $\vert f(z) \vert > 1$. However, this contradicts the Casorati-Weierstrass Theorem, which states that $f(D(z_0;\delta)\setminus \{z_0\})$ is dense in $\cc$. Thus $f$ has a pole at $z_0$.
\newpage
\subsection*{Problem 2}
Let us suppose that $\vert f(z) \vert \sim \exp(1/\vert z \vert)$ near $0$. Then, we know that
\[
\lim_{z\rightarrow 0} \vert f(z) \vert = \lim_{z\rightarrow 0} \exp(1/\vert z \vert) = \infty
\] By Problem $1$, we find that $f$ has a pole at $0$. Let us suppose that this pole is of order $k$. Then, we know that
\[
\lim_{z \rightarrow 0} z^{k+1} f(z) = 0
\] Thus, there must exist some $\delta > 0$ such that $0<\vert z\vert < \delta$ implies that
\[
\vert z^{k+1} f(z) \vert < 1
\] so that
\[
\vert f(z) \vert < \frac{1}{\vert z \vert ^{k+1}}
\] This implies that 
\[
\vert f(z) \vert \sim \frac{1}{\vert z \vert^{k+1}}
\] near $0$, which contradicts our assumption that
\[
\vert f(z) \vert \sim \exp(1/\vert z \vert)
\] near $0$ (since $\exp(1/\vert z\vert)$ approaches $\infty$ much faster than $1/\vert z \vert^{k+1}$ for any positive integer $k$). Thus no such function $f$ can exist.
\newpage
\subsection*{Problem 3}
First, we claim that $f$ is a polynomial. Since $f$ is entire, we know that it may be expressed as a power series:
\[
f(z) = \sum_{k=0}^\infty C_k z^k
\] If all but finitely many of the $C_k$ are zero, then $f$ is a polynomial. Thus, we may suppose that infinitely many $C_k$ are nonzero. Then, we have
\[
g(z) := f(1/z) = \sum_{k=0}^\infty C_k (1/z)^k = \sum_{k=-1}^{-\infty} C_{-k} z^k
\] Since there are infinitely many non-zero terms in the principal part of the Laurent expansion of $g(z)$ about $z=0$, we find that $z=0$ is an essential singularity of $g(z)$. Let $D$ be a deleted neighborhood of $0$. By the Casorati-Weierstrass Theorem, we know that $g(D)$ is dense in $\cc$. Note that $\cc \setminus \overline{D}$ is open (since $\overline{D}$ is closed). By the Open Mapping Theorem, $g(\cc\setminus \overline{D})$ is open. Thus $g(D) \cap g(\cc\setminus \overline{D}) \neq \varnothing$. That is, the image of the deleted neighborhood $D$ is dense in $\cc$ and must intersect every open set, which includes $g(\cc\setminus \overline{D})$. This informs us that there must exist points $z_1 \in D$ and $z_2 \in \cc \setminus \overline{D}$ such that $g(z_1) = g(z_2)$, so that $g$ is not injective. However, we know that $g$ must be injective since it is the composition of the injective functions $f$ and $1/z$. This contradiction implies that $f$ must be a polynomial. If $\deg f \geq 2$, then $f$ would have at least $2$ roots, contradicting the injectivity of $f$. Thus $f(z) = az+ b$ for complex constants $a$ and $b$.
\newpage
\subsection*{Problem 6}
The exponential function is never zero, so the missing value is $0$. Suppose that $\beta \neq 0$. We must find some $z$ in the annulus $A = \{z \mid 0 < \vert z \vert < 1\}$ such that $e^{1/z} = \beta$. Let $\beta = Re^{i\theta}$ for some $R > 0$ and $\theta \in [0,2\pi)$. Letting $\omega = 1/z$, we find that
\[
\re \omega = \log R
\] and
\[
\imag \omega = \theta+2\pi k
\] where $k$ is an integer. Notice that
\[
z = \frac{(\re \omega)^2} {(\re \omega)^2 + (\imag \omega)^2} - \frac{(\imag \omega)^2}{(\re \omega)^2 + (\imag \omega)^2}i
\] so that $z$ is evidently in the annulus $A$.
\newpage
\subsection*{Problem 7}
We may represent $f$ and $g$ as follows:
\[
f(z) = \frac{A(z)}{(z-z_0)^m}
\] and
\[
g(z) = \frac{B(z)}{(z-z_0)^n}
\] where $A(z_0), B(z_0) \neq 0$ and $A(z),B(z)$ are both analytic at $z_0$. First, we consider the case of $f+g$. Without loss of generality, suppose that $m \geq n$. Then we have
\begin{align*}
f(z) + g(z) &= \frac{A(z)}{(z-z_0)^m} + \frac{B(z)}{(z-z_0)^n} = \frac{A(z)}{(z-z_0)^m} + \frac{B(z)(z-z_0)^{m-n}}{(z-z_0)^n(z-z_0)^{m-n}} \\
&= \frac{A(z)}{(z-z_0)^m} + \frac{B(z)(z-z_0)^{m-n}}{(z-z_0)^{m}} = \frac{A(z) + B(z)(z-z_0)^{m-n}}{(z-z_0)^m}
\end{align*} Notice that the numerator is nonzero when $z = z_0$, so that $f+g$ has a pole of order $m$ at $z_0$. Similar reasoning holds if $n \geq m$. Thus we deduce that $f+g$ has a pole of order $\max\{m,n\}$ at $z_0$. Next, we consider the case of $f \cdot g$. We have
\[
f(z) \cdot g(z) = \frac{A(z)}{(z-z_0)^m} \cdot \frac{B(z)}{(z-z_0)^n} =  \frac{A(z)B(z)}{(z-z_0)^{m+n}}
\] so that $f \cdot g$ has a pole of order $m+n$ at $z_0$. Finally, we may consider the case of $f/g$.
 Notice that
\[
\frac{f(z)}{g(z)} = \frac{A(z)}{B(z)} \frac{(z-z_0)^n}{(z-z_0)^m}
\] If $m > n$, then $z_0$ is a pole of order $m-n$. If $n > m$, then $z_0$ is a zero of order $n-m$ of $f/g$. If $n = m$, then $z_0$ is a removable singularity.
\newpage
\subsection*{Problem 9}
\subsubsection*{Part A}
Notice that
\[
\frac{1}{z^4+z^2} = \frac{1}{z^2(z^2+1)} = \frac{1}{z^2(z+i)(z-i)}
\] Note that $0$ is a pole of order $2$. This is true because
\[
\lim_{z \rightarrow 0} z^2 \cdot \frac{1}{z^4+z^2} = \lim_{z \rightarrow 0} \frac{1}{(z+i)(z-i)} = 1
\] and
\[
\lim_{z \rightarrow 0} z^3 \cdot \frac{1}{z^4+z^2} = \lim_{z \rightarrow 0} \frac{z}{(z+i)(z-i)} = 0
\] We claim that $\pm i$ are poles of order $1$. This is true because
\[
\lim_{z \rightarrow i} (z-i) \frac{1}{z^2(z+i)(z-i)} = \lim_{z \rightarrow i} \frac{1}{z^2(z+i)} = -\frac{1}{2i}
\] while
\[
\lim_{z \rightarrow i} (z-i)^2 \frac{1}{z^2(z+i)(z-i)} = \lim_{z \rightarrow i} \frac{(z-i)}{z^2(z+i)} = 0
\] Also, we have
\[ 
\lim_{z \rightarrow -i} (z+i) \frac{1}{z^2(z+i)(z-i)} = \lim_{z \rightarrow -i} \frac{1}{z^2(z-i)} = \frac{1}{2i}
\] while
\[
\lim_{z \rightarrow -i} (z+i)^2 \frac{1}{z^2(z+i)(z-i)} = \lim_{z \rightarrow -i} \frac{(z+i)}{z^2(z-i)} = 0
\]
\subsubsection*{Part B}
Recall that 
\[
\cot z = \frac{\cos z}{\sin z}
\] The singularities are the integral multiples of $\pi$. We claim that these singularities are poles of order $1$. Notice that 
\[
\lim_{z\rightarrow k \pi} (z-k \pi) \frac{\cos z}{\sin z} = (-1)^k \lim_{z\rightarrow k \pi} (z-k \pi) \frac{\cos z}{\sin (z - k \pi)} = (-1)^k \cos(k\pi) = (-1)^{2k} = 1
\] Thus, we have
\[
\lim_{z\rightarrow k \pi} (z-k \pi)^2 \frac{\cos z}{\sin z} = (-1)^k \lim_{z\rightarrow k \pi} (z-k \pi)^2 \frac{\cos z}{\sin (z - k \pi)} = (-1)^k \lim_{z\rightarrow k \pi} (z-k \pi) \cos z = 0
\] This proves that the singularities are poles of order $1$.
\subsubsection*{Part C}
Note that
\[
\csc z = \frac{1}{\sin z}
\] The singularities are again the integral multiples of $\pi$. As in Part B, we claim that they are poles of order $1$. Notice that
\[
\lim_{z\rightarrow k\pi} (z- k\pi) \frac{1}{\sin z} = (-1)^k \lim_{z\rightarrow k \pi} (z-k \pi) \frac{1}{\sin (z - k \pi)} = (-1)^k
\] Furthermore, we have
\[
\lim_{z\rightarrow k\pi} (z- k\pi)^2 \frac{1}{\sin z} = (-1)^k \lim_{z\rightarrow k \pi} (z-k \pi)^2 \frac{1}{\sin (z - k \pi)} = (-1)^k \lim_{z\rightarrow k \pi} (z-k \pi) = 0
\] This shows that all the singularities are poles of order $1$.
\subsubsection*{Part D}
First, we claim that
\[
\frac{\exp(1/z^2)}{z-1}
\] has a pole of order $1$ at $z = 1$. Notice that
\[
\lim_{z\rightarrow 1} (z-1) \frac{\exp(1/z^2)}{z-1} = \lim_{z\rightarrow 1} \exp(1/z^2) = e 
\] and that
\[
\lim_{z\rightarrow 1} (z-1)^2 \frac{\exp(1/z^2)}{z-1} = \lim_{z\rightarrow 1} (z-1) \exp(1/z^2) = 0 
\] so that the singularity at $z = 1$ is a pole of order $1$. Next, we claim that the singularity at $z = 0$ is essential. Notice that
\[
\frac{\exp(1/z^2)}{z-1} = \sum_{k=0}^{\infty} \bigg(\frac{1}{z^2}\bigg)^k \cdot \frac{1}{k!(z-1)}
\] so that the function has infinitely many terms in the principal part of its Laurent expansion centered at $0$. Thus the singularity at $z = 0$ is essential. 
\end{document} 